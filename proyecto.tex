\documentclass[oneside,oldfontcommands]{memoir}
\usepackage{listings}
\usepackage[utf8]{inputenc}
\usepackage[spanish]{babel}
\usepackage[T1]{fontenc}
\usepackage{kpfonts}
\usepackage{float}
\usepackage{algorithm}
\usepackage{subcaption}
\usepackage[export]{adjustbox}
\usepackage{algorithmic}
\usepackage{cite}
\usepackage{variables}
\usepackage{chapterheader}
\usepackage{xspace}
\usepackage{subfiles}
\usepackage{xcolor}
\definecolor{lightblue}{HTML}{4A70AD}
\decimalpoint
\usepackage{dcolumn}
\newcolumntype{.}{D{.}{\esperiod}{-1}} \makeatletter
\addto\shorthandsspanish{\let\esperiod\es@period@code} \makeatother
\RequirePackage{verbatim}
\usepackage{graphicx}
\usepackage{afterpage}
\usepackage{longtable}
\usepackage[pdfborder={000}]{hyperref} %referencia

\hypersetup{
  pdfauthor = {\AuthorName (email (en) ugr (punto) es)},
  pdftitle = {\ProjectTitle},
  pdfsubject = {},
  pdfkeywords = {palabra_clave1, palabra_clave2, palabra_clave3, ...},
  pdfcreator = {LaTeX con el paquete ....},
  pdfproducer = {pdflatex},
  colorlinks = True,
  linkcolor = lightblue
}

\usepackage{url}
\usepackage{colortbl,longtable}

\renewcommand{\chaptermark}[1]{\markboth{\textbf{#1}}{}}
\renewcommand{\sectionmark}[1]{\markright{\textbf{\thesection. #1}}}

\setlength{\headheight}{1.5\headheight}

\newcommand{\HRule}{\rule{\linewidth}{0.5mm}}
% Definimos los tipos teorema, ejemplo y definición podremos usar
% estos tipos simplemente poniendo \begin{teorema} \end{teorema} ...
\newtheorem{teorema}{Teorema}[chapter]
\newtheorem{ejemplo}{Ejemplo}[chapter]
\newtheorem{definicion}{Definición}[chapter]

\newcommand{\bigrule}{\titlerule[0.5mm]}

% Para conseguir que en las páginas en blanco no ponga cabeceras
\makeatletter
\def\clearpage{
  \ifvmode \ifnum \@dbltopnum =\m@ne \ifdim \pagetotal <\topskip
  \hbox{} \fi \fi \fi
  \newpage
  \thispagestyle{empty} \write\m@ne{} \vbox{} \penalty -\@Mi }
\makeatother

\begin{document}
	
\subfile{prefaces/cover}

\subfile{prefaces/spanish_abstract}

\subfile{prefaces/english_abstract}

\tableofcontents

\setlength{\parskip}{5pt}
%1. Resumen y palabras clave.
%Breve resumen del trabajo realizado. Se incluirán seguidamente al menos cinco palabras
%clave que definan el trabajo a criterio del autor.
%2. Resumen extendido y palabras clave en inglés.
%Deberá estar escrito completamente en inglés y tener una longitud mínima de 1500 palabras.
%Igualmente aparecerán las palabras clave en inglés.

%4. Objetivos del trabajo.
%En este apartado deberán aparecer con claridad los objetivos inicialmente previstos en la
%propuesta de TFG y los finalmente alcanzados con indicación de dificultades, cambios y
%mejoras respecto a la propuesta inicial. Si procede, es conveniente apuntar de manera precisa
%las interdependencias entre los distintos objetivos y conectarlos con los diferentes apartados
%de la memoria.
%Se pueden destacar aquí los aspectos formativos previos más utilizados.
\subfile{chapters/01_introduction}

%Se explicarán los métodos y procesos empleados para desarrollar el trabajo y alcanzar los
%objetivos. Es conveniente destacar tanto los métodos inicialmente previstos como aquellos
%que hayan tenido que ser agregados en el desarrollo del trabajo.
%Éste es el lugar de presentar todos los datos técnicos y científicos realizados en el TFG.
%Debe ser detallado, claro y preciso. Contendrá al menos dos subapartados. Uno de ellos
%estará dedicado al aspecto matemático del trabajo, y el otro al aspecto informático.
%\subfile{chapters/02_metodologia-juego}

%\subfile{chapters/02_metodologia-patrones}

%\subfile{chapters/02_metodologia-montecarlo}

%En caso de que en el TFG se desarrolle software, se recomienda que el apartado 5 contenga
%además los siguientes puntos:
%Planificación y presupuesto. Se incluirá la planificación temporal con su correspondiente
%división en fases y tareas, y la posterior comparación con los datos reales obtenidos tras
%realizar el proyecto. También se adjuntará un presupuesto del trabajo a realizar.
%Análisis y diseño. Se incluirá la especificación de los requerimientos y la metodología de
%desarrollo por la que se ha optado, así como los “planos“ del proyecto, que contendrán las
%historias de usuario o casos de uso, diagrama conceptual, diagramas de iteración, diagramas
%de diseño, esquema arquitectónico y bocetos de las interfaces de usuario. Además, se
%describirán las estructuras de datos fundamentales y los desarrollos algorítmicos no triviales.
%Implementación y pruebas. Se incluirán todos los aspectos relacionados con la programación
%de la aplicación y las tecnologías seleccionadas, justificándolas e introduciendo el diseño de
%pruebas e informes de ejecución de las mismas.
%El código fuente del software desarrollado, ya sea como un resultado principal del proyecto o
%como un medio para la realización de simulaciones y pruebas, deberá ser incluido en la versión
%electrónica de la memoria
%\subfile{chapters/02_metodologia-implementacion} % representacion interna, facilidades y generalidades de mi programa

%\subfile{chapters/02_metodologia-flujoYtratamientoDatos}



%\subfile{chapters/03_resultados}

%Las conclusiones deberán incluir todas aquellas de tipo profesional y académico. Además,
%se deberá indicar si los objetivos han sido alcanzados totalmente, parcialmente o no
%alcanzados.
%Si hubiese posibles vías claras de desarrollo posterior sería interesante destacarlas aquí,
%poniéndolas en valor en el contexto inicial del trabajo.
%\subfile{chapters/04_conclusionesYperspectivas}


\bibliographystyle{acm}
\bibliography{Bibliography}

\end{document}
