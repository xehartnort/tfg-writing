\documentclass[../proyecto.tex]{book}

\begin{document}

\chapter{Introducción}

El autómata celular fue inventado por von Neumann y Ulam en 1950 para estudiar el problema de contruir máquinas artificiales que se reproduzcan a sí mismas \cite{neummanUlam}. Con el fin de imitar el comportamiento de los seres vivos, el diseño de dichas máquinas incluiría el espacio en el que se desarrollan representado por una malla en la que los nodos son llamados células y evolucionan simultáneamente de acuerdo a un conjunto de reglas simples. Éstas dirigen la \textit{física} de su pequeño universo abstracto y son locales en el sentido de que cada célula tiene solamente conocimiento de aquellas células que la rodean, su vecindario. En la construcción de Neumman el vecindario está constituido por las células adyacentes verticales y horizontales, lo que se conoce como vecindario de Neumman o de tipo Neumman, si además se considera que las células adyacentes diagonales forman parte del vecindario, entonces pasaría a nombrarse vecindario de Moore o de tipo Moore.

El juego de vida es un autómata celular propuesto por Conway en 1970 y popularizado por Martin Gardner en el mismo año \cite{primerap}. Éste consiste en la evolución de una disposición inicial de células con dos estados mutuamente excluyentes, vida (1) o muerte (0), en una malla rectangular infinita. Dicha evolución viene dada por un conjunto de reglas que se aplican simultáneamente a todas las células considerando el vecindario de tipo Moore. Las reglas son las siguientes: dada una célula viva, ésta continua viviendo solo si en su vecindario hay dos o tres células, en otro caso muere y dada una célula muerta, ésta renace si tiene exactamente tres células en su vecindario.

Uno de los motivos que atrajo la atención de científicos de diversos campos es la capacidad de observar como patrones complejos surgen de la aplicación de un conjunto muy simple y reducido de reglas. De esta manera comenzaron a observarse configuraciones iniciales que daban lugar a comportamientos interesantes, tales como las de naves espaciales \ref{fig:spaceship} que se desplazan sobre la malla rectangular, los osciladores \ref{fig:blinker} que retornan a su configuración inicial después de un número finito de aplicaciones de las reglas o las vidas inmóviles \ref{fig:block} que no ven alterada su forma tras la aplicación de las reglas.

\begin{figure}[H]
	\centering
	\includegraphics[height=.15\linewidth]{./images/glider.png}
	\caption{Ejemplo de nave espacial.}
	\label{fig:spaceship}
\end{figure} 
\begin{figure}[H]
	\centering
	\includegraphics[height=.125\linewidth]{./images/blinker.png}
	\caption{Ejemplo de oscilador de periodo dos.}
	\label{fig:blinker}
\end{figure} 
\begin{figure}[H]
	\centering
	\includegraphics[height=.15\linewidth]{./images/block.png}
	\caption{Ejemplo de vida inmóvil.}
	\label{fig:block}
\end{figure} 

La elección de las reglas de evolución parecería a priori aleatoria, sin embargo, Conway las escogió de acorde a las siguientes pautas \cite{libroGardner}:
\begin{itemize}
	\item No debe existir una disposición inicial de células para la cual haya una prueba simple de que la población crezca sin límite.
	\item Debe de haber disposiciones iniciales de células que aparentemente crezcan sin límite. 
	\item Debe de haber disposiciones iniciales de células simples que crezcan y cambien durante un periodo considerable, llegando a tres posibles finales: desaparecer completamente ya sea debido a superpoblación o a dispersión, estabilizarse en una configuración que se mantenga constante o entrar en un ciclo sin fin de oscilación.
\end{itemize}

Al intentar realizar simulaciones de juegos de vida, los ordenadores se erigieron como la herramienta principal para llevarlas a cabo, por tanto enfrentaron al problema de representar una malla rectangular infinita en un ordenador con memoria finita. Una inteligente solución es alterar las características topológicas de la malla rectangular, identificando los bordes opuestos para obtener superficies topologicamente equivalentes a las de una botella de Klein, una esfera, o un toro. En particular, esta última resultó atraer gran interés, pues se obtiene evidencia de que reduce los efectos asociados a la finitud de la malla \cite{finitudMalla, finitudMalla2}. Cabe también descatar el estudio de la alteración de las cualidades geométricas de la malla, tales como el uso de figuras geométricas regulares diferentes al cuadrado (triángulo y hexágono)\cite{triangular}, teselaciones de Penrose \cite{penrose} o el empleo del espacio geométrico hiperbólico \cite{hiperbolico}. Finalmente, existen implementaciones en las cuales no se almacena la malla en la memoria, si no que se almacena la posición de cada célula respecto a un origen de coordenadas, simulando más eficazmente una malla  rectangular infinita \cite{boardless}.

El juego de vida de Conway también muestra interesantes características en el campo de teoría de la computación, pertenece a la clase IV de Wolfram \cite{ccuatro, ccuatro2} y se ha demostrado que tiene la capacidad de cómputo de una máquina de Turing universal \cite{turingUniversal}. Por tanto existe una disposición inicial de células que simula una máquina de Turing, la cual fue extendida a una máquina universal de Turing \cite{turing}. Más recientemente en un esfuerzo colectivo, se ha implementado un ordenador con su propio lenguaje de ensamblador, compilador a lenguaje de alto nivel, y sobre este último se ha programado el conocido juego Tetris \cite{tetris, logical}.

Si los autómatas celulares son un modelo que representa a organismos vivos se podría pensar que la hipótesis de actualización simultánea es cuestionable, es decir, en la naturaleza no se propaga la información de manera instantánea y mucho menos de manera perfecta. Aunque haya algunos casos en los cuales el comportamiento del autómata celular permanece constante aunque se cambie la hipótesis de transferencia instantánea y perfecta de la información como expondremos más adelante en este trabajo. En esencia, estamos observando la robustez del modelo a perturbaciones en su evolución. Un modelo será robusto, si pequeños cambios en su evolución se traducirán en pequeñas perturbaciones del comportamiento global del sistema, mientras que si esta pequeña modificación produce un cambio cualitativo en la dinámica del sistema, éste será poco robusto o simplemente sensible a las variaciones, dicho cambio cualitativo se conoce también como transición de fase. En la literatura se plantean dos maneras diferentes de introducir las perturbaciones a través de la asincronicidad en la aplicación reglas de evolución \cite{asyncIntro}:
\begin{itemize}
	\item \textbf{Evolución totalmente asíncrona}: en cada unidad de tiempo discreto, las reglas de evolución se aplican solamente a un individuo escogido del conjunto de células, el criterio de elección puede ser o no ser completamente aleatorio.
	\item \textbf{Evolución $\alpha$-asíncrona}: en cada unidad de tiempo discreto, cada célula tiene probabilidad $\alpha$ de aplicar las reglas de evolución y probabilidad $1-\alpha$ de mantener su estado.
\end{itemize}

Estos esquemas de evolución también se conocen como \textbf{evolución guiada por pasos} y \textbf{evolución guiada por tiempo}, respectivamente \cite{aka}. Los primeros en estudiar los efectos de la evolución asíncrona frente a la evolución síncrona en el juego de vida fueron \cite{syncVSasync}, aplicando un esquema de evolución $\alpha$-asíncrona demostraron la existencia de una transición de fase de un comportamiento \"estático\", donde el sistema termina alcanzando alguna situación completamente estable, a un comportamiento 'vívido', y por tanto, inestable. Posteriormente se estudió como afectaban las variaciones en la topología de la malla a la transición de fase, concluyendo que el valor crítico de la transición de fase depende fuertemente de la regularidad de la malla rectángular \cite{mallaIrregular1, mallaIrregular2}.

Hasta donde podemos saber solo se ha estudiado el comportamiento en situaciones de $\alpha$-asincronicidad de configuraciones iniciales aleatorias, también conocidas como \"sopas\", con una densidad prefijada, obteniendo resultados que se comparan con las características conocidas del juego de vida síncrono. En este trabajo queremos caracterizar la manera en la que configuraciones iniciales bien conocidas y estudiadas como las que se muestran en las figuras \ref{fig:blinker}, \ref{fig:block} y \ref{fig:spaceship}, alteran su comportamiento en situaciones de $\alpha$-asincronicidad, a través de diferentes variables que midan el crecimiento de la población y su evolución, entre otras.

\end{document}

% !TeX root = ../proyecto.tex
