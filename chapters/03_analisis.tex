\documentclass[../proyecto.tex]{memoir}

\begin{document}

\chapter{Análisis}

En esta sección expondremos y comentaremos los resultados obtenidos en las simulaciones realizadas, en última instancia la tarea se verá representada en forma de comentarios a gráficos resultantes de medir las variables de interés propuestas. Este trabajo tendrá como objetivo dos niveles de concreción:

\begin{itemize}
\item El primero será la extracción de características notables que nos permitan diferenciar cual es el impacto global sobre las configuraciones iniciales de la $\alpha$-asincronicidad en el esquema de actualización del juego de vida de Conway. 

\item El segundo irá orientado a la comprobación de si las categorías de configuraciones iniciales del juego de vida de Conway expuestas con anterioridad se preservan. En caso negativo, propondremos atendiendo a los datos experimentales, nuevas clasificaciones para obtener una nueva descripción taxonómica del juego de vida de Conway $\alpha$-asíncrono que se ajuste adecuadamente.
\end{itemize}


\section{Formulación de las simulaciones}

La manera probablemente más intuitiva de realizar las simulaciones de la evolución de configuraciones iniciales del juego de vida de Conway $\alpha$-asíncrono consistiría en la ejecución de un número prefijado de simulaciones, cada una con el mismo número prefijado de pasos, para después agregar en cada paso de la evolución los datos obtenidos de cada simulación individual. De esta manera en cada paso de la evolución se tendría una visión global de los estados más probables a esta formulación la llamaremos secuencial. 

La formulación secuencial permitiría dos niveles de análisis para cada configuración inicial:

\begin{itemize}
\item Valores medios de las variables de interés en cada paso: si agregamos en cada paso los valores de las simulaciones, tras verificar su distribución sea normal, obtendremos un estimador del valor medio de cada característica de interés en cada paso. Esto es, aplicaremos los fundamentos de Monte Carlo sobre los valores agregados en cada paso de las simulaciones.

\item Valores medios de las variables de interés en la simulación: supuesto el anterior nivel realizado, es posible aplicar el hecho de que la suma de variables aleatorias que pertenecen una distribución normal, da lugar a una variable aleatoria de una distribución normal, luego sería posible aplicar los mismos principios que en el nivel anterior para obtener \textit{buenas} aproximaciones de los valores medios de la variables de interés. Estos principios serían el empleo de los intervalos de confianza de la distribución normal para indicar que el valor medio aproximado se encuentra muy probablemente dentro de un intervalo dado en función de la media de nuestro estimador y su desviación estándar.
\end{itemize}

\end{document}