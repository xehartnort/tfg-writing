\documentclass[../proyecto.tex]{memoir}

\begin{document}

\chapter{Metodología}

Los formalismos nos permitirán articular la intuición de asincronismo en el esquema de actualización del autómata celular, cuyo posible homónimo biológico sería el procesamiento imperfecto de información entre individuos a causa de perturbaciones derivadas del medio o de la interacción con otros individuos. En este trabajo nos restringimos a un caso simple de asincronismo en la actualización: examinaremos que ocurre si todas las transiciones ocurren al mismo tiempo pero los individuos reciben la información del estado de sus vecinos de forma imperfecta.

En primer lugar introducimos el autómata celular m-asíncrono que nos permite articular el juego de vida de Conway $\alpha$-asíncrono, a continuación exponemos los conceptos básicos de teoría de conjuntos, probabilidad y generación de números aleatorios sobre los que desarrollaremos las estimaciones Monte Carlo. Las claves de este desarrollo son el teorema central del límite y la ley de los grandes números que esencialmente justifican la efectividad del método Monte Carlo.
 
\section{Teoría de la computación}

\subsection{Introducción}
Dado que el comportamiento completamente síncrono de un autómata celular como herramienta de modelado es una rareza, se han realizado numerosas investigaciones empíricas del autómata celular asíncrono. Sin embargo, los pocos análisis formales realizados o bien se refieren a ejemplos o bien a casos particulares de asíncronicidad. Tomaremos el concepto de autómata celular $m$-asíncrono \cite{oraculo}, cuya idea principal es tener algún tipo de oráculo el cual en cada iteración escoge las células que tienen que ser actualizadas. Dicho oráculo se implementa a través de una medida de probabilidad $\mu$ sobre subconjuntos de enteros d-dimensionales, $\mathds{Z}^{d}$. Notar que la definición con la que trataremos es la extensión a espacios multidimensionales de la dada en \cite{oraculo}.

\subsection{Autómata celular determinista}
Un autómata celular determinista es un sistema dinámico discreto consistente en un array $d$-dimensional de autómatas finitos, llamados células. Cada célula está conectada uniformemente a un vecindario formado por un número finito de células, tiene un estado de un conjunto finito de estados y actualiza su estado de acuerdo a una función de transición local, la cual determina el siguiente estado de una célula considerando su propio estado y el de su vecindario. 

\begin{defi}
Formalmente, la tupla $A=(\mathds{Z}^{d}, N, Q, f)$ es un autómata celular determinista, de ahora en adelante autómata celular, donde:

\begin{itemize}
\item $\mathds{Z} ^{d}$ es un espacio de células $d$-dimensional.
\item $Q$ el conjunto de estados posibles para cada célula.
\item $N \in (\mathds{Z}^{d})^{k}$ es el vecindario genérico de un autómata celular, esto es, para $N=(n_{1},...,n_{k})$, $a \in \mathds{Z} ^{d}$ célula, cada célula en $\{(a+n_{1},...,a+n_{k})\}$ es una célula vecina de $a$.
\item $f:Q^{k+1} \rightarrow Q$ es la función de transición local que define la transición de estado de cada célula como función de su propio estado y del estado de cada célula en su vecindario. 
\end{itemize}

\end{defi}

\begin{defi}
Una configuración es una función $g: \mathds{Z}^{d} \rightarrow Q$, la cual a cada punto del espacio $\mathds{Z}^{d}$ le asigna un estado del conjunto de estados $Q$, al conjunto de las configuración lo notaremos $Q^{\mathds{Z}^{d}}$. Entenderemos por configuración inicial, a aquella configuración a la que aún no se le ha aplicado la función de transición global.
\end{defi}

\begin{defi}
La función de transición local induce una función de transición global $F:Q^{\mathds{Z}^{d}} \rightarrow Q^{\mathds{Z}^{d}}$ definida como sigue:
$$
\forall x \in Q^{\mathds{Z}^{d}}, \quad \forall i \in \mathds{Z}^{d}, \quad F(x)(i) = f(x(i),x(i+n_{1}),...,x(i+n_{k})).
$$
\end{defi}

Nos gustaría poder plasmar la intuición de que una célula tenga la misma probabilidad de ser actualizada independientemente o no del resto de células actualizadas en cada iteración. El concepto de ultrafiltro nos permitirá establecer una clase de autómata celular lo suficientemente general como para abarcar los nuevos tipos de autómatas celulares que introduciremos en posteriores secciones. 

Dado un conjunto X, $\mathcal{P}(X)$ denota el conjunto de todos los subconjuntos de X. Dado $A \in \mathcal{P}(X)$, notaremos su complementario $A^{c}$. 
\begin{defi}
$U \in \mathcal{P}(X)$ es un ultrafiltro de X si:

\begin{enumerate}
\item $\emptyset \in U$.
\item Sean $A,B \in \mathcal{P}(X)$ tales que $A \subset B$ y $ A \in U$, entonces $B \in U$.
\item Si $A,B \in U$, entonces $A \cap B \in U$.
\item Si $A \in \mathcal{P}(X)$ entonces o bien $A \in U$, o bien $ A^{c} \in U$.
\end{enumerate}

Además dado $p \in X$, el ultrafiltro $U_{p}$ diremos que es principal si es el más pequeño que contiene a $p$, esto es, la colección de todos los conjuntos que contienen a $p$.
\end{defi}

\begin{defi}
Un autómata celular $m$-asíncrono $C$ es una tupla $(A, \mu)$ donde: 
\begin{itemize}
\item A es un autómata celular.
\item $\mu$ es una medida de probabilidad sobre la $\sigma$-álgebra de Borel en $\mathcal{P}(\mathds{Z}^{d})$.
\end{itemize} 
\end{defi}

\begin{defi}

Para cada función de transición local $f$ y cada conjunto $\tau \in \mathcal{P}(\mathds{Z}^{d})$, definimos la función de transición global $F:Q^{\mathds{Z}^{d}} \to Q^{\mathds{Z}^{d}}$ como sigue:

\begin{equation*}
	\forall x \in Q^{\mathds{Z}^{d}}, \quad \forall i \in \mathds{Z}^{d}, \qquad
	F_{\tau}(x)(i) = \left\{ \begin{array}{lcc}
             f(x(i),x(i+n_{1}),...,x(i+n_{k})) &   si  & i \in \tau ,\\
             \\ x(i) & si  & i \notin \tau .\\
             \end{array}
             \right.
\end{equation*}

Es decir, $F_{\tau}$ aplica la función de transición local solo sobre los elementos de $\tau \subset \mathds{Z}^{d}$. 
\end{defi}

Notar que cada célula $i \in \mathds{Z}^{d}$ es actualizada con probabilidad $\mu(U_{i})$.

Esta nueva definición de autómata celular $m$-asíncrono, incluye la de autómata celular síncrono. Fijada una $\sigma$-álgebra $\mathds{B}$ sobre $\mathds{Z}^{d}$ y sea $C_{0}=(A, \mu_{0})$ un autómata celular $m$-asíncrono donde $\mu_{0}: \mathds{B} \rightarrow [0,1]$ viene dada por: 

\begin{equation*}
	 \forall A \in P(\mathds{Z}^{d}), \qquad 
	 \mu_{0}(A) = \left\{ \begin{array}{lcc}
             \ 1 &   si  & \mathds{Z}^{d} \in A ,\\
             \\0 &   si  & \mathds{Z}^{d} \notin A .\\
             \end{array}
             \right.
\end{equation*}

De esta manera, $\mu_{0}(\{\mathds{Z}^{d}\})=1$ y por lo tanto, en cada instante de tiempo se aplicará la función de transición local sobre $\mathds{Z}^{d}$.

Por otro lado, también contiene el concepto de evolución totalmente asíncrona comentado en la introducción, esto es, en cada instante se aplica la función de transición local a una sola célula. 

\begin{defi}
Consideramos ahora el autómata celular $m$-asíncrono $C=(A, \mu_{1})$ donde $\mu_{1}: \mathds{B} \rightarrow [0,1]$ verifica lo siguiente:

\begin{enumerate}
\item $\mu_{1}(U_{i}) > 0, \quad \forall i \in \mathds{Z}^{d}$.
\item $\mu_{1}(U_{i} \cap U_{j}) = 0, \quad \forall i \neq j, \quad i,j \in \mathds{Z}^{d}$.
\end{enumerate}
\end{defi}

Así solo los ultrafiltros de la forma $\{k\}$ $(k \in \mathds{Z}^{d})$ se les aplica la función de transición local.

Por último, contiene el concepto de evolución $\alpha$-asíncrona que nos interesa. 
\begin{defi} \label{alhpaasin}
Dado $C=(A, \mu_{2})$ un autómata celular $m$-asíncrono y sea $\alpha \in (0,1)$ la probabilidad con la que se actualizan las células, donde $\mu_{2}: \mathds{B} \to [0,1]$ satisface:

\begin{enumerate}
\item $\mu_{2}(U_{i}) = \alpha, \quad \forall i \in \mathds{Z}^{d}$.
\item $ \forall A \subseteq \mathds{Z}^{d} \quad$ finito,$\quad  \mu_{2} ( \bigcap_{a \in A} U_{a} ) = \prod_{a \in A} \mu_{2} ( U_{a} )$.
\end{enumerate}
\end{defi}

y lo notaremos $C(\alpha)$.
\subsection{Juego de vida de Conway}

\begin{defi}  \label{original}
El juego de vida de Conway es un autómata celular síncrono: $$
C = (\mathds{Z}^{2}, N, Q, f)
$$
donde
\begin{itemize}
\item $N=\{(-1, 1), (0, 1), (1, 1), (-1, 0), (1, 0), (-1,-1), (0,-1), (1,-1) \}$,
\item $Q=\{0,1\}$,
\item $f:\{0,1\}^{9} \rightarrow \{0,1\} $ dada por:

\begin{equation} \label{trans}
f(x)= \left\{ \begin{array}{lcc}
             1 &   si  & x_{0}=0 \quad y \quad \sum_{i=1}^{8} x_i = 3 \\
             \\ 1 & si & x_{0}=1 \quad y \quad \sum_{i=1}^{8} x_i \in \{2 ,3\} \\
             \\ 0 &  si  & \sum_{i=1}^{8} x_i \notin \{2, 3\} \
             \end{array}
   \right. 
\end{equation}
y $x = (x_{0}, x_{1}, ...,x_{8}) = (c,c+n_{1},...,c+n_{8})$ con $c \in \mathds{Z} ^{d}$ célula.

\end{itemize}
\end{defi}

\subsection{Juego de vida de Conway $\alpha$-asíncrono}
\begin{defi}
El juego de vida de Conway $\alpha$-asíncrono es un autómata celular $m$-asíncrono formado por $C=(\mathds{Z}^{2}, N, Q, f)$, el autómata celular síncrono definido anteriormente \ref{original}, y una medida de probabilidad $\mu: \mathds{B} \rightarrow [0,1]$ verificando las condiciones expuestas en \ref{alhpaasin} para $\alpha \in (0,1)$. 
\end{defi}


\section{Teoría de la probabilidad}

El contenido de esta sección está extraído de los siguientes textos: \cite{elLibro, grandesNumeros, loeve}. 

\begin{defi}
Una $\sigma$-álgebra, $\mathds{F}$, sobre un conjunto $X$, es una colección no vacía de subconjuntos de $X$ cerrados para uniones numerables y para la operación de complementario, esto es:
\begin{itemize}
\item $\forall A \in \mathds{F}$ se verifica que $A^{c} \in \mathds{F}$.
\item $ \forall A_{n} \in \mathds{F}, n \in \mathds{N} $ se verifica que $\bigcup _{n \in \mathds{N}} A_{n} \in \mathds{F}$.
\end{itemize}
\end{defi}

\begin{defi}
Sean un conjunto $X$ con su $\sigma$-álgebra asociada, $\mathds{F}$, el par $(X, \mathds{F})$ es un espacio medible. 
\end{defi}

\begin{defi}
Una función medible es una función entre espacios medibles, $g: (X, \mathds{F}) \rightarrow (X', \mathds{F}')$ tal que: $g^{-1}(A) \in \mathds{F} \quad \forall A \in \mathds{F}'$.
\end{defi}

\begin{defi}
La tupla $(X, \mathds{F}, P)$ es un espacio de probabilidad si:
\begin{itemize}
\item $X$ es el espacio de muestreo, esto es, algún conjunto no vacío.
\item $\mathds{F}$ es una $\sigma$-álgebra de sucesos.
\item $P: \mathds{F} \to \mathds{R}$ es una medida de probabilidad, 
esto es, $P$ satisface los siguientes axiomas de Kolmogorov:
\begin{enumerate}
\item Para cada $A\in\mathds{F}$, existe un número $P(A) \geq 0$, esto es, la probabilidad del suceso $A$,
\item $P(X)=1$.
\item Sean ${A_n, n \geq 1}$ disjuntos, entonces: $$
	P \left( \bigcup_{n=1}^{\infty} A_{n} \right) = \sum_{n=1}^{\infty} P(A_n).
$$
\end{enumerate}
\end{itemize}
\end{defi}

\begin{defi}
Los sucesos ${A_n, n \geq 1}$ son independientes si y solo si $$
P \left( \bigcap_{n \geq 1} A_{n} \right) = \prod_{n \geq 1} P(A_n).
$$
\end{defi}

\begin{defi}
Un conjunto $A$ es abierto si, para cada punto $x\in A$, existe una bola de centro el punto y radio $\epsilon > 0$, $B(x,\epsilon)=\{ z : \abs{z-x} < \epsilon\}$, tal que $B(x,\epsilon) \subset F$. Así la $\sigma$-álgebra de Borel, es aquella generada por los conjuntos abiertos de $\mathds{R}$.
\end{defi}

De ahora en adelante supondremos que se ha fijado el espacio medible dado por $X=\mathds{R}$ y $F=\mathds{B}$ la $\sigma$-álgebra de Borel sobre $\mathds{R}$.

\subsection{Variables aleatorias}

\begin{defi}
Una variable aleatoria definida sobre un espacio de probabilidad $(X, \mathds{B}, P)$ es una función medible $A: X \to \mathds{R}$.
\end{defi}

Cada valor de $A$ se corresponde con un subconjunto de puntos de $X$ que se aplica en dicho valor: $\{ w\in X : A(w)=x\}$, que notaremos por simplicidad $\{X = x\}$. A parte de los anteriores conjuntos también nos resultarán de interés lo siguientes:
\begin{align*}
\{ w\in X : A(w) \leq x\} = \{ A \leq x \} \\
\{ w\in X : A(w) < x\} = \{ A < x \} \\
\{ w\in X : A(w) > x\} = \{ A > x \} \\
\{ w\in X : A(w) \geq x\} = \{ A \geq x \}
\end{align*}
%Con esta útil notación podremos notar fácilmente la probabilidad de sucesos de mayor interés.

\begin{prop} \label{funcion_de_va}
Si $g:\mathds{R}\to\mathds{R}$ es medible y $A$ es una variable aleatoria entonces $A'=g(A)$ es una variable aleatoria.
\end{prop}

\begin{defi}
La variables aleatorias $A_1, A_2,..., A_n$ son independientes si y solo si, para arbitrarios conjuntos de la $\sigma$-álgebra de Borel $B_1, B_2,..., B_n$: $$
	P \left( \bigcap_{k = 1}^{n} \{A_k \in B_k\} \right) = \prod_{k = 1}^{n} P(A_k \in B_k).
$$
\end{defi}

\begin{defi}
Dada una variable aleatoria $A$ se define su función de distribución como $ F : \mathds{R} \to [0,1] $ dada por:
$$
x \mapsto F(x)=P(A \leq x).
$$
\end{defi}

\begin{prop}
La función de distribución de la variable aleatoria $A$ satisface:
\begin{itemize}
\item Es monótona no decreciente.
\item Dada una sucesión decreciente de elementos de $\mathds{R}$, $\{x_n\}_{n \in \mathds{N}} \in \mathds{R}$ convergente a $x\in \mathds{R}$ se tiene $\lim_{x_n \to x} F(x_n) = F(x)$, es decir, es continua a la derecha.
\item $\lim_{x\to+\infty} F(x) = 1$ y $\lim_{x\to-\infty} F(x) = 0$.
\end{itemize}
\end{prop}

\begin{defi}
Sea $F$ una función de distribución definimos la función de densidad como la función integrable, $f$, tal que:
$$
F(b)-F(a) = \int^{b}_a f(x) dx, \quad \forall a < b. 
$$
\end{defi}

\begin{defi}
Sea $A$ una variable aleatoria, definimos su valor esperado o esperanza como sigue:
$$
\mathds{E}(A) = \int_{X}A(w)dP(w).
$$
Adicionalmente si $\mathds{E}\abs{A} < \infty$, diremos que $A$ es integrable.
\end{defi}

\begin{defi}
Sea una variable aleatoria $A$, definimos:
\begin{itemize}
\item Los momentos de orden $n$ de $A$: $\mathds{E}(A^n) = \int_{X}A(w)^{n}dP(w), \quad n \in \mathds{N}$.
\item Los momentos centrados de orden $n$ de $A$: $\mathds{E}_c(A^n) = E( (A-E(A))^n ), \quad n \in \mathds{N}$.
\end{itemize}
\end{defi}

Notar que los momentos no existen necesariamente para todo $n\in \mathds{N}$.

\begin{defi}
Definimos la varianza de la variable aleatoria $A$ con esperanza $\mu < \infty$ y $\mathds{E}_c(A^2) < \infty $ como:$$
var(A) \equiv \mathds{E}_c(A^2) = \mathds{E}(A^2) - \mu^2.
$$
\end{defi}

\begin{defi}
A la raíz cuadrada positiva de la varianza la notaremos $\sigma(A)=+\sqrt{var(A)}$ y diremos que es la desviación estándar de la variable aleatoria A. 
\end{defi}

\begin{prop}
Sea $A$ una variable aleatoria con esperanza $\mu < \infty$ y varianza $\sigma^2 < \infty$ y sea $A'=aA+b$, donde $a,b\in\mathds{R}$, entonces: $$
\mathds{E}(A')=a\mu + b \quad y \quad var(A') = a^2 \sigma^2.
$$

\end{prop}


\subsection{Variables aleatorias discretas}

\begin{defi}
Una variable aleatoria $A$ diremos que es discreta si toma valores es un conjunto numerable, esto es, $\exists E=\{x_n\}_{n \in \mathds{N}} \subset \mathds{R}$ tal que $P(A \in E)=1$. 
\end{defi}

\begin{defi}
La función de distribución de una variable aleatoria discreta $A$ es la siguiente: $$
\forall x\in \mathds{R}, \quad F(x) = P( A \leq x) = \sum_{x_n\in E, x_n \leq x} P(A=x_n).
$$
\end{defi}

\begin{defi}
Sea una variable aleatoria discreta $A$, definimos:

\begin{itemize}
\item Los momentos de orden $n$ de $A$: $\mathds{E}(A^n) = \sum_{x_m \in E} x_m^n P(A=x_m)$.
\item Los momentos centrados de orden $n$ de $A$: $\mathds{E}_c(A^n) =\sum_{x_m \in E} (x_m - \mu)^n P(A=x_m)$.
\end{itemize}
\end{defi}

El momento $n=1$ se conoce como valor esperado o esperanza: $$
\mathds{E}(A) \equiv \sum_{x_m \in E} x_m P(A=x_m).
$$

\begin{defi}[Tipos de convergencia: convergencia en probabilidad, convergencia casi segura y convergencia en distribución]
Sean ${A_n}$ ,$n\in \mathds{N}$ y $A$ variables aleatorias, definimos:
\begin{itemize}
\item $A_n \to A$ en probabilidad, si para todo $\epsilon > 0$, $\lim_{n\to\infty} P( |A_n-A|> \epsilon ) = 0$ y lo notaremos $A_n \to^{P} A$.
\item $A_n \to A$ casi seguramente, si $P(\lim_{n\to\infty} A_n=A) = 1$ y lo notaremos $A_n \to^{c.s.} A$.
\item $A_n \to A$ en distribución, si $\lim_{n \to \infty} P(A_n \leq x) = P(A \leq x),\quad \forall x \in \mathds{R}$ donde $x\mapsto P(A \leq x)$ es una función continua y lo notaremos $A_n \to^{d} A$.
\end{itemize}
\end{defi}

\begin{defi}
Sea $i$ el número completo $i=\sqrt{-1}$, la extensión de la función exponencial $exp: \mathds{R} \to \mathds{R^{+}}$ al cuerpo de los números complejos es $exp: \mathds{C} \to \mathds{C}$ dada por:
$$
z \mapsto exp(iz) = e^{iz}=\cos z+i\sin z.
$$
\end{defi}

Notar que el módulo de la exponencial compleja está acotado por la unidad :

$$
\abs{ exp(iz) } = \abs{\cos z+i\sin z} = \sqrt{\cos^2 z + \sin^2 z} = 1.
$$

Esta propiedad de la exponencial compleja nos asegura la existencia para toda variable aleatoria de la siguiente definición.

\begin{defi}
La función característica asociada a la variable aleatoria $A$ es la función $\phi_{A}: \mathds{R} \to \mathds{C}$, dada por:
$$
\phi_{A}(t) = \mathds{E}(e^{itA}).
$$
\end{defi}

\begin{prop}
Sea $\phi_A$ la función característica de la variable aleatoria $A$, entonces:

\begin{itemize}
\item $\abs{\phi_A(t)} \leq \phi_A(0)=1$.
\item $\phi_A$ es una función uniformemente continua.
\item $\phi_{cA+b}(t)=e^{itb}\phi_A(ct), \quad c,b\in \mathds{R}$.
\item Sea $A_1, A_2,...,A_n, n\in\mathds{N}$ una sucesión finita de variables aleatorias independientes, entonces: $$ 
\phi_{\sum^{n}_{i=1} A_i} (t) = \prod_{i=1}^{n} \phi_{A_i} (t) .
$$ Además si $A_1, A_2,...,A_n$ son idénticamente distribuidas: $$
\phi_{\sum^{n}_{i=1} A_i} (t) = \left( \phi_{A_1}(t) \right)^{n}.
$$
\item Si $\mathds{E}(A^k) < \infty$ su derivada k-ésima evaluada en 0 es $\phi_A^{(k)}(0)=i^k\mathds{E}(A^k)$.
\end{itemize}

\end{prop}

\subsection{Distribución normal}

\begin{defi} \label{normal}
Diremos que una variable aleatoria $A$ sigue una distribución normal con media $\mu < \infty$ y varianza $\sigma^2 < \infty$, $N(\mu,\sigma^2)$, si $A$ tiene la siguiente función de densidad: 

$$
f(x) = \frac{1}{ \sigma \sqrt{ 2 \pi }} exp \left( -\frac{1}{2}\left( \frac{x-\mu}{\sigma} \right)^2 \right),\quad \forall x \in \mathds{R}.
$$

Además su función característica es: $$
 \phi_{A}(t)=e^{-\frac{t^2}{2}}.
$$
\end{defi}

\begin{prop} 
Sean $A, A'$ variables aleatorias pertenecientes a distribuciones normales con medias $\mu,\mu' < \infty$ y varianzas $\sigma_{0}^2,\sigma_{1}^{2} < \infty$, $N(\mu,\sigma_{0}^{2})$ y $N(\mu',\sigma_{1}^{2})$ respectivamente, entonces la variable aleatoria dada por la suma $A+A'$ es una variable aleatoria que pertenece a una distribución normal $N(\mu + \mu', \sigma_{0}^{2}+\sigma_{1}^{2} )$.
\end{prop}

\begin{prop} \label{prop_normal}
Sea $N(\mu,\sigma^2)$ una distribución normal con media $\mu<\infty$ y varianza $\sigma^2<\infty$, entonces el 68.27\% de los valores de la distribución normal se encuentran en el intervalo $[\mu+\sigma, \mu-\sigma]$, el 95.45\% en el intervalo $[\mu+2\sigma, \mu-2\sigma]$ y el 99.7\% en el intervalo $[\mu+3\sigma, \mu-3\sigma]$, estos intervalos se conocen también como intervalos de confianza de la distribución normal.
\end{prop}

\subsection{Teorema central del límite}

\begin{teorema} \label{central}
Sean $A_{1},A_{2},...,A_{n}$ variables aleatorias independientes, con esperanza $\mu < \infty$, varianza $\sigma^2 < \infty$ e idénticamente distribuidas. Entonces: $$
\frac{1}{ \sigma \sqrt{n}} \left( \frac{1}{n}\sum_{i=1}^nA_i - n\mu \right) \to^d N(0,1).
$$
\end{teorema}

Previa a la demostración del teorema central del límite, introducimos las herramientas matemáticas que nos harán posible su demostración.

\begin{teorema}[Teorema de Taylor]

Sea $k \in \mathds{N}$ y $f: \mathds{R} \to \mathds{R}$ k-veces diferenciable en el punto $a \in \mathds{R}$. Entonces existe una función $h_k: \mathds{R} \to \mathds{R}$ tal que:

$$
f(x)=f(a)+f'(a)(x-a)+\frac{f''(a)}{2!}(x-a)^2+\dotsb+\frac{f^{(k)}(a)}{k!}(x-a)^k + h_k(x)(x-a)^k
$$

y $\lim_{x\to a} h_k(x) = 0$.
\end{teorema}

\begin{teorema}[Teorema de continuidad] \label{cont}
Sean $A_1, A_2,...,A_n, n\in\mathds{N}$ variables aleatorias entonces: $$
\lim_{n \to \infty }{\phi_{A_n}} = \phi_{A}(t), \quad \forall t\in \mathds{R},
$$
si y solo si $$
A_n \to^d A.
$$

\end{teorema}

\begin{proof}[Demostración (Teorema central del límite)]

Sean $A_{1},A_{2},...,A_{n}$ variables aleatorias independientes idénticamente distribuidas con esperanza $\mu < \infty$ y varianza $\sigma^2< \infty$.
Sea ahora $$
Z_{n} = \frac{A_{1}+A_{2}+...+A_{n}-n\mu }{\sigma \sqrt{n}}.
$$
Definimos una nueva variable aleatoria, $Y_i$, que es la versión normalizada de $A_i$:$$
Y_i=\frac{A_i-\mu}{\sigma}
$$
Así definida, $Y_i$ es idénticamente distribuida con esperanza y varianza: $$
E(Y_{i}) = 0,  Var(Y_{i})=1.
$$
Sea ahora $Z_n = \frac{Y_1+Y_2+...+Y_n}{\sqrt{n}}$, queremos ver que: $$
\lim_{n \to \infty} \phi_{Z_{n}}(t)=e^{-\frac{t^{2}}{2}}.
$$

Procedemos a desarrollar el siguiente término:

$$
\phi_{\frac{Y_1+Y_2+...+Y_n}{\sqrt{n}}}(t) = \prod_{i=1}^{n} \phi_{\frac{Y_i}{\sqrt{n}}} (t) = \big(\phi_{\frac{Y_1}{\sqrt{n}}}(t) \big)^n.
$$

Aplicando el teorema de Taylor para obtener el desarrollo centrado en 0 para $k=2$ de $\phi_{Y_1}(\frac{t}{\sqrt{n}})$: 

\begin{align*}
\phi_{\frac{Y_1}{\sqrt{n}}}(t) &= \phi_{Y_1}(\frac{t}{\sqrt{n}}) \\
  &= \phi_{Y_1}(0) + \frac{t}{\sqrt{n}}\phi_{Y_1}^{'}(0) + \frac{t^2}{2n}\phi_{Y_1}^{''}(0) + \frac{t^2}{n} h_2(t)\\
 &= 1 + i\frac{t}{\sqrt{n}}\mathds{E}(Y_1) - \frac{t^2}{2n}\mathds{E}(Y_1^2) + \frac{t^2}{n} h_2(t)\\
 &= 1 + 0 - \frac{t^2}{2n} + \frac{t^2}{n} h_2(\frac{t}{\sqrt{n}}),
\end{align*}
donde $\lim_{t\to 0} h_2(\frac{t}{\sqrt{n}}) = 0$.

Obtenemos que: $$
\big(\phi_{Y_1}(\frac{t}{\sqrt{n}}) \big)^n =\big( 1 - \frac{t^2}{2n} + \frac{t^2}{n} h_2(\frac{t}{\sqrt{n}}) \big)^n\longrightarrow e^{-\frac{t^2}{2}}.
$$

cuyo límite es la función característica de una variable aleatoria perteneciente a una distribución normal con media 0 y varianza 1, concluimos la demostración aplicando el teorema de continuidad \ref{cont}.
\end{proof}

\subsection{Ley de los grandes números}
\begin{teorema} \label{teo_grandes_numeros}
Sean $A_n$, $n \in \mathds{N}$ variables aleatorias independientes, con esperanza $\mu < \infty$ e idénticamente distribuidas. Entonces el valor medio de $A_n$, $\bar{\mu}$, converge casi seguramente a $\mu$:

$$
\bar{\mu}=\frac{1}{n}\sum_{n\in\mathds{N}} A_n \to^{c.s.} \mu,
$$

esto es, $P(\lim_{n\to\infty} \bar{\mu}=\mu) = 1$.

\end{teorema}

Para obtener una demostración un tanto más breve de éste teorema añadiremos la hipótesis de existencia del momento de orden 4 de $A_n$. Una demostración completa del teorema sin la hipótesis adicional se puede consultar en \cite{elLibro}.

\begin{lema}  \label{intercambio_suma}
Sean $A_n$,$ n \geq 1$ variables aleatorias no negativas, entonces: $$
E \left( \sum_{n \geq 1} A_n \right) = \sum_{n \geq 1} E(A_n).
$$

\end{lema}

\begin{lema}
En las misma condiciones de la ley de los grandes números, existe una constante $ K < \infty $ tal que para todo $ n \geq 0$:
$$
\mathds{E} \big( ( \bar{ \mu } - n \mu ) ^ 4 \big) \leq K n^2.
$$
\end{lema}

\begin{proof}

Sean 
$$
Z_k = A_k - \mu \quad y \quad T_n = Z_1 + Z_2 + ... + Z_n = \sum_{i=1}^{n} A_n - n\mu. 
$$

Entonces:

$$
\mathds{E} ( T_{n}^{4} ) = \mathds{E} ( \big( \sum_{i=1}^{n} Z_i \big) ^{4} ) = n\mathds{E}(Z_{1}^4)+3n(n-1)\mathds{E}(Z_1^2 Z_2^2) \leq Kn^2.
$$

donde en la segunda igualdad se ha empleado el desarrollo multinomial:

$$
(x_1+x_2+...+x_m)^n = \sum_{k_1+k_2+...+k_m=n} { n \choose k_1,k_2, ..., k_n} \prod_{1 \leq t \leq m} x_t^{k_t},
$$

con

$$
{ n \choose k_1,k_2, ..., k_n} = \frac{n!}{k_1!k_2! \dotsb k_n!}.
$$

Dado que $\mathds{E}(Z_k)=0 \quad \forall k$ y la independencia de las variables $Z_k$, se cancelan todos los sumandos de la forma:

\begin{align*}
	\mathds{E} (Z_{i} Z_{j}^3 ) &=\mathds{E} (Z_{i}) \mathds{E} (Z_{j}^3) = 0, \quad 1 \leq i,j \leq n, \quad i \neq j, \\
	\mathds{E} ( Z_{i} Z_{j} Z_{k} Z_{l} ) &= \mathds{E} (Z_{i}) \mathds{E} (Z_{j}) \mathds{E} (Z_{k}) \mathds{E} (Z_{l}) = 0, \quad 1 \leq i,j,k,l \leq n, \quad i \neq j \neq k \neq l.
\end{align*}

y siendo $K$ adecuadamente elegida, por ejemplo $K = 4 max\{\mathds{E}(Z_1^4),\ \mathds{E}(Z_1^2)^2\}$.
\end{proof}

Ya tenemos todos los rudimentos necesarios para proceder a demostrar el teorema de esta sección.

\begin{proof}[Demostración (Ley de los grandes números)]

Asumamos que $\mathds{E}_c( A_n^4) < \infty \quad \forall n$, aplicando el lema anterior:

$$
	\mathds{E} \big( ( \bar{ \mu } - \mu ) ^ 4 \big) \leq \frac{K}{n^{2}}.
$$

Ahora sea $Y_n = ( \bar{ \mu } - \mu ) ^ 4, \forall n \in \mathds{N}$ una variable aleatoria por la proposición \ref{funcion_de_va} y en particular es no negativa, luego podemos aplicar el lema \ref{intercambio_suma} en la siguiente cadena de igualdades:

$$
	\mathds{E} \big( \sum_{n \geq 1}( \bar{ \mu } - \mu ) ^ 4 \big) = \sum_{n \geq 1} \mathds{E} \big( ( \bar{ \mu } - \mu ) ^ 4 \big)  \leq K\sum_{n \geq 1}\frac{1}{n^{2}} < \infty,
$$

lo que implica:

$$
\abs{ \sum_{n \geq 1} \big( \bar{ \mu } - \mu \big)^ 4 } = \sum_{n \geq 1} \big( \bar{ \mu } - \mu \big)^ 4 < \infty \quad c.s. 
$$

Pero si una serie es convergente, entonces la sucesión de su término general de la serie converge a cero, por tanto:
$$
 \bar{ \mu } \to \mu \quad c.s.
$$
\end{proof}


\section{Fundamentos de las simulaciones Monte Carlo} \label{MonteCarlo}

El nombre \textit{Monte Carlo} fue acuñado por los científicos que trabajaban en el desarrollo de armas nucleares en Los Álamos en la década de los 40 para designar una clase de métodos numéricos basados en el uso de números aleatorios. La esencia de este método reside en la invención de juegos de azar cuyo comportamiento puede ser usado para estudiar algún fenómeno de interés. Se podría pensar que el hecho de que resultados obtenidos por estos métodos estén sujetos a las leyes del azar es un problema, sin embargo, es un problema menor puesto que se puede determinar como de exactos son sus resultados y si se deseara obtener resultados más precisos, bastaría con incrementar el número de experimentos realizados. Actualmente, los métodos de Monte Carlos juegan un papel fundamental en la resolución de problemas matemáticos complejos, en los cuales, o bien los métodos de resolución analíticos o bien los métodos numéricos existentes requieren de grandes periodos de tiempo cómputo.

Curiosamente, en las definiciones de los métodos de Monte Carlo no hay referencia explícita al empleo de la capacidades de cómputo de los ordenadores, sin embargo el gran desarrollo que han experimentado éstos desde el último tercio de siglo XX hasta nuestros días, los ha convertido en herramientas indispensables en las simulaciones Monte Carlo. La generación de números aleatorios ha experimentado también un importante crecimiento en las últimas décadas.

\subsection{Generadores de números pseudo-aleatorios}

Los generadores de números pseudo-aleatorios producen secuencias indistinguibles de una realmente aleatoria, es decir, no generan valores de una distribución uniforme si no que dado una semilla o valor de inicialización generan siempre la misma sucesión de números.

Un tipo generador de números aleatorios muy usado venía dado por la siguiente ecuación expresión:

\begin{equation} \label{cong}
I_{j+1} = aI_{j} +c \mod (m)
\end{equation}

donde $a$ es un entero positivo llamado multiplicador y $c$ es un número natural llamado incremento. Para $c \neq 0$, el generador de la ecuación \ref{cong} es conocido por el nombre de \textit{generador lineal congruente}. Claramente, en $n<m$ pasos la ecuación comienza a generar valores duplicados en el mismo orden. Conocido ésto, se hacían elecciones particulares de $a,c$ y $m$ que proporcionaran el mayor periodo posible. En \cite{knuth} podemos encontrar algunos resultados notables sobre la elección parámetros $a,c$ y $m$. La elección del valor inicial $I_{0}$ no es relevante, pues se generarán todos los naturales posibles entre $0$ y $m-1$ antes de la primera repetición. Sin embargo no es suficiente con generar una sucesión de números de un largo periodo, además deben superar rigurosas baterías de tests empíricos que aseguren una buena distribución de las secuencias, además de la ausencia de patrones en las mismas. En el caso de los generadores lineales congruentes existe un resultado que afirma que las sucesivas n-tuplas de valores generados residen en al menos $(n!m)^\frac{1}{m}$ hiperplanos paralelos \cite{marsagliaRandom}, luego esta clase de generadores no son adecuados para la generación de números aleatorios.

Dado el carácter empírico de las baterías de test, superarlas con éxito no asegura un generador de números perfecto, sino que probablemente se trate de un buen generador, ya que eventualmente con el suficiente tiempo se podría encontrar un test que no fuera superado con éxito. Por tanto nos interesaremos en tests que demuestren el mal comportamiento de un generador en un tiempo razonable. 

De entre las numerosas baterías de tests existantes destacamos dos \cite{dieharder,testu01}: \textit{Dieharder}, que está basada en los primeros test estadísticos propuestos en \textit{Diehard battery of tests}, incluye también los test desarrollados por el NIST (National Institute for Standards and Technology) y la batería de tests \textit{TestU01} \cite{dieharder,testu01}.

Nuestra elección es una variante del generador \textit{Mersenne Twister} \cite{mt} presente en la biblioteca estándar del lenguaje de programación Python en las versiones posteriores a la $2.3$ \cite{pyver}, y en el lenguaje de programación estadística $R$\cite{langR}.

\subsection{Métodos Monte Carlo} \label{carlino}

Los métodos de Monte Carlo se apoyan fundamentalmente en dos grandes resultados de la teoría de la probabilidad: la ley de los grandes números y el teorema central de límite. Ambos nos permiten describir la distribución límite de la suma de las variables aleatorias independientes, proporcionando también una estimación del error. 

\begin{defi}
Una muestra aleatoria simple $S_n$, es un conjunto de $n\in\mathds{N}$ variables aleatorias, $A_1,A_2,...,A_n$, independientes e idénticamente distribuidas. En caso de que la media y la varianza de las variables aleatorias $A_1,A_2,...,A_n$ sean finitas, las notaremos $\mu$ y $\sigma^2$ respectivamente.
\end{defi}

La media de una muestra aleatoria simple $S_n$, $\mu_S = E(S_n) = \frac{1}{n}\sum_{i=1}^n A_i$ es una variable aleatoria gracias al resultado \ref{funcion_de_va}, si además la media $\mu$ y varianza $\sigma$ de $A_i$ son finitas, dicha variable aleatoria tiene la siguiente media y varianza:$$
E(\mu_S) = E(\frac{1}{n}\sum_{i=1}^n A_i) = \frac{1}{n}\sum_{i=1}^n E(A_i) = \mu,
$$
$$
var(\mu_S) = var(\frac{1}{n}\sum_{i=1}^n A_i) = \frac{1}{n^2} \sum_{i=1}^n var(A_i) = \frac{\sigma^2}{n}.
$$
Luego su desviación estándar es $\sqrt{var(\mu_S)} = \frac{\sigma}{\sqrt{n}}$.
\begin{defi}
La variable aleatoria $\mu_S$ anteriormente definida diremos que es un \textit{estimador} del valor esperado $E(A)=\mu$.
\end{defi}

\begin{defi} \label{chebi}
Sea $\mu_S$ un estimador de la media de una muestra aleatoria simple, $S_n$, de media $\mu < \infty$ y varianza $\sigma^2  < \infty$ y $\delta>0$, la desigualdad de Chebychev es:

$$
P \left( \abs{\mu_S-\mu} \geq \sqrt{\frac{var(\mu_S)}{\delta}} \right) = P \left( \abs{\mu_S-\mu} \geq \frac{\sigma}{\sqrt{n\delta}} \right)  \leq \delta.
$$

\end{defi}

El gran resultado o \textit{teorema fundamental de las simulaciones Monte Carlo} que podemos deducir de los anteriores, en particular es una consecuencia directa del teorema \ref{teo_grandes_numeros}, el estimador $G$ de la media de una muestra aleatoria simple, $S_n$, de media $\mu < \infty$ y varianza $\sigma^2 < \infty$ converge en probabilidad al valor esperado $\mu$:
$$
\forall \epsilon > 0,\quad \lim_{n\to\infty} P( |G - \mu|> \epsilon ) = 0.
$$

Es posible aplicar la desigualdad \ref{chebi} para obtener la velocidad de convergencia respecto a $n$. Veamos un ejemplo de éste hecho, dado $\delta=\frac{1}{100}$:

$$
P \left( ( G - \mu )^2 \geq \frac{100}{n} \sigma^2 \right)  \leq \frac{1}{100},
$$

Haciendo $n$ lo suficientemente grande, la varianza de $G$ se hace tan pequeña como se quiera, esto es, disminuye considerablemente la probabilidad de obtener una gran desviación relativa a $\delta$ entre el valor esperado y el los valores obtenidos.

Es posible obtener un resultado más fuerte que el anterior como consecuencia  del teorema \ref{central}. Existe una función de distribución de probabilidad que aproxima los valores del estimador $G$, esto es, cuando $n \to\infty$, el teorema central del límite afirma que asintóticamente los valores de $G$ convergen a una distribución normal \ref{normal}. Por tanto, es posible reescribir la función de distribución como sigue:
$$
f(G) = \sqrt{\frac{n}{2 \pi \sigma^2}} exp \left( - \frac{n(G-\mu)^2}{2\sigma} \right).
$$
Cuando $n \to\infty$, el valor de $G$ se encuentra en intervalos cada vez más estrechos centrados en $\mathds{E}(G)$ y es posible medir la desviación en unidades de $\sigma$, es decir, el valor de $G$ está dentro del intervalo centrado en $\mathds{E}(G)$ de un error estándar el 68.3 \% de las veces, de dos errores estándar el 95.4 \% de las veces y de tres errores estándar el 99.7 \%  de las veces, los intervalos son $[\mathds{E}(G)+\sqrt{var(G)},\ \mathds{E}(G)-\sqrt{var(G)}]$, $[\mathds{E}(G)+2\sqrt{var(G)},\ \mathds{E}(G)-2\sqrt{var(G)}]$ y $[\mathds{E}(G)+3\sqrt{var(G)},\ \mathds{E}(G)-3\sqrt{var(G)}]$, respectivamente \ref{prop_normal}. Como comentábamos anteriormente la convergencia es asintótica por lo que inicialmente desconocemos como de grande debe de ser $n$ para poder aplicar el teorema.

Cuando la varianza no es finita, es posible encontrar una distribución límite para $G$ que llevará a un caso particular del teorema central del límite, en estos casos la distribución límite no será en general la distribución normal. Un estimador de la varianza de la media estimada viene dado por:
$$
var(G_n) = \frac{1}{n-1} \left( \frac{1}{n} \sum_{i=1}^n A_i^2 - \left( \frac{1}{n} \sum_{i=1}^n A_i \right)^2 \right)
$$

\end{document}
