\documentclass[../proyecto.tex]{memoir}

\begin{document}

\chapter{Conclusiones}

%Hasta donde alcanza nuestro conocimiento solo se ha estudiado de manera general cómo se modifica el comportamiento de configuraciones aleatorias pero no se ha estudiado cómo cambian las diferentes clases de configuraciones en el juego de vida $\alpha$-asíncrono, lo que ha motivado que en este trabajo desarrollemos su análisis.
En este trabajo se ha estudiado la evolución $\alpha$-asíncrona de tres tipos de configuraciones del juego de vida de Conway, mediante las técnicas de simulación Monte Carlo.

Una de las cuestiones más interesantes que se ha observado en todas las ejecuciones es que los promedios de las variables medidas alcanzan un valor aproximadamente constante para cada valor escogido de $\alpha$, independientemente del número de iteraciones. A pesar de que podría introducir inestabilidad, el $\alpha$-asíncronismo induce cierta \textit{estabilidad} en las variables observadas, concluyendo que las configuraciones reproducen un esquema de auto-organización, hecho que ya se había puesto de manifiesto para configuraciones iniciales aleatorias. 

%Expresar esto en "positivo"
A menudo se ha observado que cuando $\alpha$ se aproxima a la unidad es habitual que los valores medios estudiados se aproximen a los valores que se obtienen en un esquema de actualización síncrono, mientras que por lo contrario cuando $\alpha$ se aproxima a cero los comportamientos son diversos y en general se alejan del comportamiento síncrono.

Otra observación de interés es que cuando $\alpha$ decrece se tiende a ralentizar el número de cambios de estados en los nodos y que por el contrario cuando $\alpha$ se aproxima a la unidad la actividad crece. Este fenómeno se ve reflejado en el comportamiento global del calor medio sobre las configuraciones estudiadas, lo que confirma la intuición de que cuanto menor sea la probabilidad de actualizar los nodos, menor será el cambio de estado de los mismos.

En general se puede observar que de las categorías propuestas de configuraciones, solo se preservan las vidas inmóviles debido a su carácter estable. Por otra parte, hemos observado  la evolución $\alpha$-asíncrona de las \textit{naves espaciales} estudiadas se puede caracterizar por la ausencia de vidas inmóviles puesto que eso no ocurre en los otros tipos de configuraciones iniciales estudiadas.
%Si por el contrario, éstos crecieran o disminuyeran para algún valor fijado de $\alpha$ se tendría que o bien la configuración crece indefinidamente o bien la configuración colapsa, estando formada únicamente por nodos vacíos. 

Como trabajo futuro, proponemos analizar más configuraciones de cada categoría, añadir categorías nuevas que permitan recoger comportamientos más complejos así como un tratamiento estadístico más completo. También atrae nuestro interés la profundización del trabajo a través de la observación de más variables que sean computacionalmente más costosas. Por ejemplo, se puede realizar un censo de los clústeres de configuraciones más frecuentes, las cuales nos permitan realizar una extrapolación del censo utilizado en éste trabajo a la situación de actualización $\alpha$-asíncrona.

% (dar un ejemplo) [que se podría añadir como resultado preliminar en la presentación]


%H \\
%Además también visitamos brevemente el campo de la generación de números pseudo-aleatorios, pues no es una tarea trivial su generación y a menudo se encuentra en lenguajes de programación de alto nivel implementaciones deficientes de los mismos.
%to the best of our knowlegde

\end{document}