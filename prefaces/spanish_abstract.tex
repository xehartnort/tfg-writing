\documentclass[../proyecto.tex]{memoir}
\begin{document}
\thispagestyle{empty}

\begin{center}
  {\large\bfseries \ProjectTitle}\\
\end{center}
  \begin{center}
  	
  \AuthorName\\
  \vspace{0.7cm}
  \noindent{\textbf{Palabras clave}: }\\

  \vspace{0.7cm}
  \noindent{\textbf{Resumen}}\\

El juego de vida de Conway ha sido uno de los autómatas celulares más estudiados desde que von Neumman y Ulam introdujeran el concepto de autómata celular. El conjunto de reglas que determinan la evolución de éste autómata celular sobre un malla son muy simples, sin embargo dan lugar a comportamientos complejos y difíciles de predecir que han atraído la atención de diversos científicos de otros campos. Desde entonces, a parte de descubrir y catalogar dichos comportamientos, también ha surgido interés en el estudio de como afectan las perturbaciones en la malla y en la evolución. Además se han propuesto nuevos tipos de autómatas celulares similares al juego de vida pero que muestran comportamientos totalmente diferentes.


\end{center}

\newpage
\end{document}
