\documentclass[../proyecto.tex]{memoir}
\begin{document}
\thispagestyle{empty}

\begin{center}
  {\large\bfseries \ProjectTitle}\\
\end{center}
  \begin{center}
  	
  \AuthorName\\
  \vspace{0.7cm}
  \noindent{\textbf{Palabras clave}: }\\

  \vspace{0.7cm}
  \noindent{\textbf{Resumen}}\\
\end{center}
El juego de vida de Conway ha sido uno de los casos más estudiados desde que von Neumman y Ulam introdujeran el concepto de autómata celular. El conjunto de reglas que determinan la evolución de éste autómata celular sobre un malla son muy simples, sin embargo dan lugar a comportamientos complejos y difíciles de predecir, que han atraído la atención de científicos de diversos campos. Desde entonces, a parte de descubrir y catalogar dichos comportamientos, también ha surgido interés en el estudio de cómo afectan las perturbaciones en la malla y en la evolución. Además se han propuesto nuevos tipos de autómatas celulares similares al juego de vida pero que presentan comportamientos totalmente diferentes, mostrando que pequeñas alteraciones en las reglas pueden llevar a grandes cambios. \\

En este trabajo hemos explorado el juego de vida de Conway, mostrando especial interés en la clasificación de los comportamientos complejos que se dan en el mismo y cuál es su evolución cuando se exponen a perturbaciones. En particular nos hemos preguntado qué ocurre si las perturbaciones se introducen a nivel de aplicación de las reglas, es decir, qué es lo que ocurre cuando las reglas no se aplican sobre todos los individuos simultáneamente, si no en función de cierta probabilidad $\alpha$. Esta modificación define los denominados autómatas $\alpha$-asíncronos. Con el fin de definir un contexto que agrupe tanto esta idea como la del juego de vida original hemos tomado la definición de autómata celular $m$-asíncrono. \\

Dado el carácter probabilístico del juego de vida $\alpha$-asíncrono, la herramienta escogida para estudiar el desarrollo de las configuraciones ha sido la técnica de simulación Monte Carlo. Éstas se apoyan principalmente en la generación de números pseudo-aleatorios de una determinada distribución y en dos teoremas fundamentales de la teoría de la probabilidad: el teorema central del límite y la ley de los grandes números. Por tanto, en este trabajo enunciamos y demostramos las versiones \textit{clásicas} de estos teoremas. \\

En este trabajo hemos concluido que en situación de evolución $\alpha$-asíncrona, las configuraciones escogidas muestran un esquema de auto-organización, independientemente del número de iteraciones. Adicionalmente, concluimos que la $\alpha$-asíncronicidad ralentiza en general la evolución de las configuraciones y que cuando $\alpha$ se aproxima a la unidad, los promedios de las variables observadas se aproximan de manera continua a lo valores que se obtienen en el juego de vida síncrono. 

\newpage
\end{document}
